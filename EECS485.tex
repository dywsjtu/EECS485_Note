\documentclass{article}
\usepackage[utf8]{inputenc}
\usepackage{url}
\setlength{\oddsidemargin}{0.0in}
\setlength{\textwidth}{6.5in}
\setlength{\topmargin}{-0.5in}
\setlength{\textheight}{9.0in}
\addtolength{\parskip}{8pt}
\title{EECS485_Note}
\author{Yinwei}
\date{December 2019}

\begin{document}

\maketitle

\section{Introduction}
\large{\textbf{C/C++ Allocation}}\\
• Static (global) and automatic
(local) variable allocation is
managed by the compiler\\
• Deallocated automatically\\
• Dynamic variable allocation is
managed by the programmer\\
• Deallocated manually\\ \\ 
\large{\textbf{Python Object Allocation}}\\
• Objects are allocated on
assignment in a private heap\\
• Objects are deallocated
automatically\\• Objects are allocated on
assignment in a private heap
• Objects are deallocated
automatically\\ \\ 
\large{\textbf{Reference counting and Garbage collecting}} \\
Deallocate when references  == 0 in every cycle\\ \\ 
\textbf{Special Case}\\
x = {} # x is a dictionary\\
y = {} # x is a dictionary\\
x["y"] = y # x references y\\
y["x"] = x # y references x\\
x = None\\
y = None
\section{Web Basics}
$<ul>$ un number list\\
$<li>$ list item \\
HTML: Hyper Text Markup Language\\
\url{https://www.w3schools.com/html/html_intro.asp }\\
CSS: Cascading Style Sheet\\
\url{https://www.w3schools.com/html/html_css.asp}\\
\large{\textbf{Hypertext}}\\
Text with embedded links to other documents.\\
$<a $ $href="/eecs/etc/events/cseevents.html">\\
Events\\
</a>$\\
\large{\textbf{Escape strings in URLS}}\\
\&$lt$ for less than \%$20$ for space\\
\large{\textbf{Content, Presentation, Layout}}\\
• HTML has tags for all.\\
• $<h1>$ is content\\
• $<b>$ is presentation \\
• Good to separate these.\\
• Content is data; presentation is words, tables, organization; layout is
visual.\\
• Use HTML for content and presentation, add CSS for layout.\\
\large{\textbf{URL Encoding}}\\
\textbf{Protocol}\\
\textbf{protocol}://server:port/path?query#fragment\\
Example: unencrypted http\\
curl --verbose http://cse.eecs.umich.edu/ \\
* Connected to cse.eecs.umich.edu (141.212.113.143) port 80 (\#0)\\
Example: encrypted https\\
curl --verbose https://cse.eecs.umich.edu/ * \\ Connected to cse.eecs.umich.edu (141.212.113.143) port 443 (\#0)\\
* TLS 1.2 connection using \url{TLS_ECDHE_RSA_WITH_AES_256_GCM_SHA384} \\
* Server certificate: www.cse.umich.edu \\
\textbf{Server}\\
protocol://\textbf{server}:port/path?query#fragment\\
help locate the machine\\
 curl --verbose http://\textbf{cse.eecs.umich.edu}/ \\
* Connected to cse.eecs.umich.edu (\textbf{141.212.113.143}) port 80 (\#0)\\
• DNS lookup translates server name into an IP address \\
Example: \\
host cse.eecs.umich.edu\\
cse.eecs.umich.edu has address 141.212.113.143 \\
\textbf{Port}\\
protocol://server:\textbf{port}/path?query#fragment\\
Port is used to identify a specific service\\
One host machine, several servers\\
80 is typically used HTTP, 443 for HTTPS\\
To check if a port is open\\
nc -v -z cse.eecs.umich.edu 80 \\
cse.eecs.umich.edu [141.212.113.143] 80 (http) open\\
To see what ports are open\\
nmap cse.eecs.umich.edu\\
PORT STATE SERVICE \\
22/tcp open ssh \\
80/tcp open http \\
443/tcp open https\\
5666/tcp open nrpe \\
Usually, it's not a good idea to port scan, some website might blacklist you for doing this.\\
\textbf{Path}\\
protocol://server:port/\textbf{path}?query#fragment\\
• path is a file name relative to the server root \\
• Default is /index.html \\
curl \url{http://cse.eecs.umich.edu} \\
is the same as\\
curl \url{http://cse.eecs.umich.edu/index.html}\\
• A different path loads a different page\\
curl \url{http://cse.eecs.umich.edu/eecs/faculty/csefaculty.html}\\
\textbf{Absolute Path vs. Relative Path}\\
\textbf{Query}\\
protocol://server:port/path\textbf{?query}#fragment\\
• query string is a general-purpose set of parameters that the server
(or specified resource on server) can use as it pleases\\
\url{http://cse.eecs.umich.edu/eecs/etc/fac/CSEfaculty.html?match=Lecturer}\\
\textbf{Fragment}\\
protocol://server:port/path?query\textbf{#fragment}\\
fragment is identified at the client, ignored by server\\
example: navigate directly to the section labeled "Linking" \url{http://en.wikipedia.org/wiki/World_Wide_Web#Linking}\\ \\
\large{\textbf{HTTP}\\
• Hypertext Transfer Protocol \\
• Request/response protocol \\
Client (your browser) opens connection to server and writes a request\\
Server responds appropriately \\
Connection is closed \\ \\
• Server can't open connection to client \\
• Completely stateless \\
• Each request is treated as brand new \\
• No state $\Rightarrow$ no history
\end{document}
